\documentclass{article}

\usepackage{graphicx}
\usepackage{wrapfig}
\usepackage{algorithmicx}
\usepackage{algpseudocode}
\usepackage{multicol}
\usepackage{amsmath}
\usepackage{multirow}
\usepackage{hyperref}
\usepackage{caption}
\usepackage{subcaption}
\usepackage{xcolor}
\usepackage[normalem]{ulem}
\usepackage[style=ieee]{biblatex}

\addbibresource{ref.bib}

\title{July2023 CSE300 Week9 Online Evaluation}
\author{Ajmain Yasar Ahmed Sahil}
\date{\today}

\begin{document}
\maketitle
\section*{DNA vs RNA Comparison}
\begin{figure}[htbp]
    \centering
    \begin{subfigure}{0.45\textwidth}
        \centering
        \includegraphics[width=\textwidth]{images/DNA.png}
        \caption{\textit{Deoxyribonucleic acid}}
    \end{subfigure}
    \hfil
    \begin{subfigure}{0.45\textwidth}
        \centering
        \includegraphics[width=\textwidth]{images/RNA.png}
        \caption{\textit{Ribonucleic acid}}
    \end{subfigure}
\end{figure}
\section*{SPEC CPU Benchmark}
\begin{table}[htbp]
    \centering
    \caption{SPECINTC2006 benchmark programs.}
    \begin{tabular}{|c|ccc|}
        \hline
        \multirow{2}{*}{Name} & \multicolumn{3}{c|}{Details}                                                                     \\ \cline{2-4} 
                              & \multicolumn{1}{l|}{Description}       & \multicolumn{1}{c|}{Instruction Count} & Reference Time \\ \hline
        gcc                   & \multicolumn{1}{l|}{GNU C Compiler}    & \multicolumn{1}{c|}{794 G}             & 8050 sec       \\ \hline
        h264avc               & \multicolumn{1}{l|}{Video Compression} & \multicolumn{1}{c|}{3793 G}            & 22130 sec      \\ \hline
    \end{tabular}
    \caption*{This table is taken from \cite{1}.}
\end{table}
Table 1 shows some of the SPEC CPU benchmark programs that we can use to assess
the execution performance of different computational processors.
\section*{Newton’s Law of Universal Gravitation}
Newton’s law of universal gravitation states that any two particles in this universe attract
each other with a force that is proportional to the product of their masses and inversely
proportional to the square of the distance between their centers. This is one of the founda-
tional laws in classical physics that was first described and formulated in Sir Isaac Newton’s
famous \textit{Principia} book \cite{2}.
\\

The equation takes the form $F_g = G\frac{m_1m_2}{r^2}$ or $F_{ab} = -G\frac{m_am_b}{|r_{ab}|^2}\hat{r}_{ab}$ among many other
forms.
\section*{Display This Section Too!}
You need to display this section in your output PDF file. This section contains information
that may help you prepare the bibliography section.
\begin{enumerate}
    \item \textbf{Citation about SPEC CPU Benchmark}
    \begin{itemize}
        \item \textbf{Authors:} David A. Patterson, John L. Hennessy
        \item \textbf{Title:} Computer Organization and Design MIPS Edition
        \item \textbf{Publisher:} Morgan Kaufmann
        \item \textbf{Year:} 2013
        \item \textbf{Edition:} 5th
    \end{itemize}
    \item \textbf{Citation about Universal Gravitation}
    \begin{itemize}
        \item \textbf{Author:} Isaac Newton
        \item \textbf{Title:} Philosophiae Naturalis Principia Mathematica
        \item \textbf{Publisher:} Jussu Societatis Regiae ac Typis Josephi Streater
        \item \textbf{Year:} 1687
    \end{itemize}
\end{enumerate}
\printbibliography
\end{document}

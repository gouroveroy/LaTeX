\documentclass{article}

\usepackage{graphicx}
\usepackage{wrapfig}
\usepackage{algorithmicx}
\usepackage{algpseudocode}
\usepackage{multicol}
\usepackage{amsmath}
\usepackage{multirow}
\usepackage{hyperref}
\usepackage{caption}
\usepackage{subcaption}
\usepackage{xcolor}
\usepackage[normalem]{ulem}
\usepackage[style=ieee]{biblatex}

\addbibresource{ref.bib}

\title{An Introduction to Stirling’s Number of the
Second Kind}
\author{ChatGPT \& S. H.}
\date{\today}

\begin{document}
\maketitle
\section{Introduction}
In combinatorics, \textit{Stirling’s number of the second kind} $S(n, k)$ is the number
of ways to \underline{partition a set of $n$ elements into $k$ non-empty subsets} \cite{1}. These
numbers arise in various areas of mathematics and have applications in \textbf{set
theory, number theory, and even computer science}\footnote{\url{https://en.wikipedia.org/wiki/Stirling_numbers_of_the_second_kind}}.

Stirling’s numbers of the second kind can be defined recursively and have
many interesting properties, which we will explore in this document.

\section{Properties of Stirling Numbers}
\subsection{Definition}
\textbf{The Stirling number of the second kind}, denoted by $S(n, k)$, is defined
as the number of ways to divide a set of n elements into $k$ non-empty subsets.
It can be written recursively as $S(n, k) = k \cdot S(n - 1, k) + S(n - 1, k - 1)$, for
$n > 0$, with the boundary conditions $S(0, 0) = 1,\ \ \  S(n, 0) = 0$ for $n > 0$,
and\ \ \  $S(n, k) = 0$ for $k > n$.

\subsection{Combinatorial Interpretation}
Stirling numbers of the second kind have a natural combinatorial interpreta-
tion. They count the ways to partition a set of n elements into $k$ non-empty
subsets. For example, consider the set $\{1, 2, 3\}$. The number of ways to
partition this set into two subsets is given by $S(3, 2) = 3$. These partitions
are:
\begin{itemize}
    \item $\{1\}, \{2, 3\}$
    \item $\{2\}, \{1, 3\}$
    \item $\{3\}, \{1, 2\}$
\end{itemize}

\subsection{Closed Form}
Stirling numbers of the second kind can be described by the following equa-
tion:
\begin{equation*}
    S(n, k) = \frac{1}{k!} \sum_{j=0}^{k} (-1)^{k-j} \binom{k}{j} j^n
\end{equation*}

\subsection{First Few Examples}
Table 1 shows the values of the Stirling numbers of the second kind, $S(n, k)$,
for small values of $n$ and $k$:

\begin{table}[htbp]
    \centering
    \begin{tabular}{|c|c|c|c|c|c|}
        \hline
        $n \backslash k$ & 1 & 2 & 3 & 4 & 5 \\
        \hline
        1 & 1 &  &  &  &  \\
        2 & 1 & 1 &  &  &  \\
        3 & 1 & 3 & 1 &  &  \\
        4 & 1 & 7 & 6 & 1 &  \\
        5 & 1 & 15 & 25 & 10 & 1 \\
        \hline
    \end{tabular}
    \caption{Stirling Numbers of the Second Kind for $n \leq 5$.}
\end{table}

\subsection{Conclusion}
\textcolor{red}{\sout{“Don’t forget to practice more problems involving Stirling numbers to fully
understand their applications!”}}

\printbibliography

\end{document}
